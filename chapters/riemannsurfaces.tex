\chapter*{Riemann Surfaces and Algebraic Curves}

\section{Algebraic Curves}

\section{Riemann Surfaces}

The relationships between algebraic curves and Riemann surfaces are embodied by
the following two theorems \cite{Griffiths89}.

\begin{theorem} \label{thm:normalization}
  {\bf (Normalization Theorem.)} For any irreducible algebraic curve $C
  \subset \PP{2}\CC$ there exists a compact Riemann surface $X$ and
  a holomorphic mapping
  \[
      \sigma : X \to \PP{2}\CC,
  \]
  such that $\sigma( X ) = C$ and $\sigma$ is injective on the
  inverse image of the set of smooth points of $C$.
\end{theorem}

A Riemann surface together with the mapping $\sigma$ is called the {\it
  normalization of $C$}. Loosely speaking, the normalization theorem states
that an algebraic curve is a Riemann surface except at the singular points.

Conversely, every compact Riemann surface can be represented by an algebraic
curve.
\begin{theorem} \label{thm: repr-theorem}
  Any compact Riemann surface $X$ can be obtained through the normalization of
  a certain plane algebraic curve $C$ with at most ordinary double points. That
  is, there exists a holomorphic mapping
  \[
      \sigma : X \to \PP{2}\CC
  \]
  such that $\sigma(X)$ is an algebraic curve possessing at most ordinary
  double points.
\end{theorem}
Many of the geometric algorithms presented in this document are designed to
avoid singular points. Except, for example, when we want to integrate a 1-form
along a path leading to a singular point in which case we ``unwrap'' the
singularity using Puiseux series. This is discussed in more detail in the
following section. However, because of this we use the terms ``curve'' and
``Riemann surface'' interchangably.
