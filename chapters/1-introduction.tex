%%%%%%%%%%%%%%%%%%%%%%%%%%%%%%%%%%%%%%%%%%%%%%%%%%%%%%%%%%%%%%%%%%%%%%%%%%%%%%%
\chapter{Introduction} \label{ch:Introduction}
%%%%%%%%%%%%%%%%%%%%%%%%%%%%%%%%%%%%%%%%%%%%%%%%%%%%%%%%%%%%%%%%%%%%%%%%%%%%%%%



% ------------------------------------------------------------------------------
\section{The Story} \label{sec:the-story}
% ------------------------------------------------------------------------------

[INSERT SETUP HERE]

\begin{figure}
  \centering
  \begin{subfigure}[b]{0.5\textwidth}
    \includegraphics[width=\textwidth]{images/livekp.jpg}
    \caption{\^{I}le de R\'{e}, France}
    \label{fig:ile-de-re}
  \end{subfigure}

  \begin{subfigure}[b]{0.5\textwidth}
    \centering
    \includegraphics[width=0.96\textwidth]{images/sd-harbor-model.jpg}
    \caption{Model of San Diego Harbor}
    \label{fig:san-diego-harbor}
  \end{subfigure}
  \label{fig:real-life}
\end{figure}

Before continuing with the story it is important to mention that KP-like waves
certainly appear in nature. In Figure \ref{fig:real-life} we see two examples of
the periodic structure emerging from chaotic conditions. The photograph in
Figure \ref{fig:ile-de-re} was captured on a particularly windy day off the
coast of \^{I}le de R\'{e} in France. Figure \ref{san-diego-harbor} was actually
taken in the lab where a conference room-sized model of the proposed San Diego
harbor was constructed [TODO REF] before direct numerical simulation was
feasible. In this image we can see two phases of waves: on the bottom left and
top right there are long, KdV-like one-dimensional waves. In the bottom right is
the characteristic hexagonal wave structure of a ``typical'' KP solution. We
will later see that these structures correspond to different genera of
quasi-periodic KP solution.

The path to quasi-periodic KP solutions begins in complex analysis with
algebraic curves. An algebraic curve is the complex solution set to a bi-variate
polynomial equation,
\begin{equation}
  C_f: \left\{
    (\lambda, \mu) \in \CC^2
    \; \big| \;
    f(\lambda, \mu) = 0, f \in \CC[x,y]
  \right\}.
\end{equation}
Algebraic curves are closely related to Riemann surfaces in that, modulo minor
details, a holomorphic mapping takes every algebraic curve to a Riemann surafce
and vice-versa. This close connection opens up a world of tools from the geomery
of Riemann surfaces to be realized using computable algebraic tools.


% ------------------------------------------------------------------------------
\section{Outline} \label{sec:outline}
% ------------------------------------------------------------------------------

Chapter \ref{ch:background} presents a succinct introduction to the field of
complex algebraic geometry and Riemann surfaces. A student approaching the
subject need only the basics of complex analysis to get started. It begins with
an introduction to algebraic curves and their geometry. We then connect these
curves to the theory of Riemann surfaces and demonstrate, at least in the
compact and connected case, that the two are synonymous. The bulk of the chapter
dives deep into the study of Riemann surfaces from this perspective of algebraic
curves building up the necessary machinery needed to define the {\it Jacobian}
of a Riemann surface and its associated {\it Period Matrix}; central objects of
focus for much of the work in this thesis as well as current research in the
field. The chapter concludes with two key objects defined on the Jacobian, the
{\it Abel Map} and {\it Riemann Constant Vector}, as well the {\it Riemann theta
  function}. Each of these are primary ingredients in the connection to and
calculation of solutions to the Kadomtsev-Petviashvili equation.

If Chapter \ref{ch:background} presents the theory then Chapter
\ref{ch:abelfunctions} contains the corresponding algorithms which allow us to
experiment with Riemann surfaces on a computer. For each key concept described
in the previous chapter we present efficient algorithms for their computation
along with a collection of examples. These algorithms are gathered together in a
Sage package called {\it Abelfunctions}. We spend some time in this chapter
discussing the high-level design of this package including how important
principles in software design allows the package to be easily extendible. 

Chapters \ref{ch:kp} and \ref{ch:determinantal} showcase the use of these
algorithms to compute the finite-genus solutions to the Kadomtsev-Petviashvili
equation as well as determinantal representation of algebraic curves. Previous
approaches to computing periodic solutions to KP are either too restrictive (the
resulting solution space is not dense in the space of periodic solutions) or
rely on direct numerical simulation where the accuracy of the solution degrades
over time. In contrast, the finite-genus solutions derived by the Riemann
surface machinery is valid for all space and time. The work on determinantal
representations presented here is the computational realization of the theory
developed by Helton and Vinnikov, the importance of which lies in the
application to polynomial optimization algorithms.

% ------------------------------------------------------------------------------
\section{Acknowledgements} \label{sec:acknowledgments}
% ------------------------------------------------------------------------------

First and foremost I would like to thank my wife, Megan Karalus. Thank you for
your love and unending patience. You stood by me when things got tough,
celebrated with me when things got better, and encouraged me to finish what I
started. Thank you to the many people in the Applied Mathematics department at
the Univeristy of Washington for shaping me into the mathematican, computer
scientist, and human being that I am today. Bernard for his mentorship and
comraderie, Randy for encouraging me to keep one toe in the HPSC world, Lauren
for her friendship and advice; and Daniel, Lowell, and Alan for their friendship
and the many hours spent at the College Inn. Thank you to Amazon AI for
believing that I can make a difference in this world by creating great things
and by providing a rich an supportive environment where the interface between
mathematical research and high performance computation find a natural home.
Finally, this work was generously supported bpy the National Science Foundation
under grant NSF-DMS-1008001. Any opinions, findings, and conclusions or
recommendations expressed in this material are those of the authors and do not
necessarily reflect the views of the funding sources.
