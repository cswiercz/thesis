\chapter{Introduction - A Connection Between Non-linear Waves and Algebraic Curves}

\section{The Kadomtsev-Petviashvili Equation}

The primary goal of this thesis is to efficiently compute a large class of
periodic solutions, the so-called {\it finite genus solutions}, to the
Kadomtsev-Petviashvili (KP) equation
\begin{equation} \label{eqn:kp}
  3u_{yy} = \frac{\partial}{\partial x} \big(
  4u_t - ( 6uu_x + u_{xxx} )
  \big),
\end{equation}
where $u = u(x,y,t)$ describes the surface height of a two-dimensional shallow
water wave. These finite genus solutions, as we will see, have a deep and
fascinating connection to the theory of complex algebraic curves and Riemann
surfaces.

The KP equation appears in various physical applications such as plasma physics
[XXX] and the study of water waves [XXX]. For example, the KP equation can be
derived by taking shallow water, long-wavelengths, weakly two-dimensional
approximations of solutions to the Euler equation [XXX].

A major and important discovery due to Zakharov and Shabat [XXX] is that the KP
equation can be written as a Lax pair [XXX]. Let $\Psi = \Psi(x,y,t)$ be an
auxiliary wave function. The KP equation is equivalent to $\Psi$ satisfying the
linear system of equations,
\begin{align}
  \Psi_y &= \Psi_{xx} + u\Psi, \label{eqn:lax-pair-a} \\ \Psi_t &= \Psi_{xxx} +
  \tfrac{3}{2}u \Psi_x + \tfrac{3}{4}(u_x + w)\Psi. \label{eqn:lax-pair-b}
\end{align}

%% To see this, we begin by enforcing sufficient smoothness of $\Psi$ and
%% therefore have it satisfy the compatibility condition $\Psi_{yt} =
%% \Psi_{ty}$. Rewriting $\Psi_{yt}$ and $\Psi_{ty}$ solely in terms of
%% $x$-derivatives of $\Psi$ and grouping by derivative order we obtain
%% \begin{align} \label{eqn:compatibility-expansion}
%%   \Psi_{yt}
%%   &=
%%   \Psi_{5x} + \tfrac{5}{2}u \Psi_{xxx} +
%%   \left( \tfrac{15}{4} u_x + \tfrac{3}{4} w \right)\Psi_{xx} + \notag \\
%%   &=
%%   \left( 3u_{xx} + \tfrac{3}{2} w_x + \tfrac{3}{2} u^2 \right)\Psi_x +
%%   \left( u_t + \tfrac{3}{4} u_{xxx} + \tfrac{3}{4} w_{xx} + [XXX]
%%   \tfrac{3}{4}u_x + \tfrac{3}{4} w \right) \Psi,
%% \end{align}
%% and
%% \begin{align}
%%   \Psi_{yt}
%%   &=
%%   \Psi_{5x} + \tfrac{5}{2}u \Psi_{xxx} +
%%   \left( \tfrac{15}{4} u_x + \tfrac{3}{4} w \right)\Psi_{xx} + \notag \\
%%   &=
%%   \left( 3u_{xx} + \tfrac{3}{2} w_x + \tfrac{3}{2} u^2 \right)\Psi_x +
%%   \left( u_t + \tfrac{3}{4} u_{xxx} + \tfrac{3}{4} w_{xx} + [XXX]
%%   \tfrac{3}{4}u_x + \tfrac{3}{4} w \right) \Psi.
%% \end{align}
%% Examining the order...[XXX]

The KP equation admits a large family of quasiperiodic solutions of the form
\begin{equation} \label{eqn: kpsol}
  u(x,y,t) = 2 \partial_x^2 \log \theta(Ux+Vy+Wt+z_0, \Omega) + c,
\end{equation}
where $\theta$ is the Riemann theta function.

\begin{definition}[Riemann theta function] \label{def: riemanntheta}
  The {\bf Riemann theta function} $\theta: \CC^g \times \hh_g \to \CC$
  is defined in terms of its Fourier series:
  \begin{equation} \label{eqn: riemanntheta}
    \theta(z,\Omega) = \sum_{n \in \ZZ^g}
    e^{2 \pi i \left( \tfrac{1}{2} n \cdot \Omega n + n \cdot z \right)}.
  \end{equation}
  This function converges absolutely and uniformly on compact sets in
  $\CC^g \times \hh_g$ where $\hh_g$ is the space of all {\it ``Riemann
    matrices''} --- complex symmetric matrices with positive definite
  imaginary part.
\end{definition}


\section{Overview}

Chapter \ref{ch:riemannsurfaces} provides the necessary algebreo-geometric
background to this topic. An undergraduate-level understanding of complex
analysis is required. Chapter \ref{ch:algorithms} 
