%%%%%%%%%%%%%%%%%%%%%%%%%%%%%%%%%%%%%%%%%%%%%%%%%%%%%%%%%%%%%%%%%%%%%%%%%%%%%%%
\chapter{Application: Finite-Genus Solutions of the Kadomtsev--Petviashvili
  Equation}\label{ch:kp}
%%%%%%%%%%%%%%%%%%%%%%%%%%%%%%%%%%%%%%%%%%%%%%%%%%%%%%%%%%%%%%%%%%%%%%%%%%%%%%%

The Kadomtsev--Petviashvili (KP) equation is a natural two-dimensional
generalization of the Korteweg de-Vries (KdV) equation describing
two-dimensional shallow water wave propagation.

\begin{figure}
  \centering
  \begin{tikzpicture}[descr/.style={fill=white}]
    \matrix(m)[matrix of math nodes, row sep=6em, column sep=12em, text
    height=1.5ex, text depth=0.25ex] {
      u(x,y,0) & X,\DivD \\
      u(x,y,t) & \boldsymbol{U},\boldsymbol{V},\boldsymbol{W},\boldsymbol{D}, \Omega \\
    };

    \path[->]
      (m-1-1) edge node[auto] {Deconinck} (m-1-2)
      (m-1-1) edge node[auto] {} (m-2-1)
      (m-1-2) edge node[near start,right]
                   {A) Krichever, Deconinck, van Hoeij}
                   node[near end, right]
                   {B) Frauendiener, Klein}
                   (m-2-2)
      (m-1-2) edge[dashed] node[descr] {Bobenko} (m-2-1)
      (m-2-2) edge[bend left,dashed] node[auto]
              {Dubrovin, Flickinger, Segur}
              (m-2-1)
      (m-2-2) edge node[auto] {Krichever} (m-2-1);
  \end{tikzpicture}
  \caption{A map of contributions to the solution to the initial value problem
    for the KP equation. The path taken by this work is represented by the solid
    arrows. Alternate approaches are shown as dashed arrows.}
\end{figure}



Other approaches exist for computing with Riemann surfaces. Bobenko and
collaborators \cite{belokolos, BobenkoBordag89} compute solutions of integrable
equations using a Schottky group representation for the associated surface. To
our knowledge, the only paper dealing with all Riemann surfaces represented by
algebraic curves is by Frauendiener, Klein, and Shramchenko who compute the
homology of a Riemann surface from the monodromy of an underlying algebraic
curve, following \cite{dvh1}. Otherwise, authors have restricted themselves to
specific families of Riemann surfaces such as hyperelliptic ones
\cite{FrauendienerKlein06,FrauendienerKlein15} or low genus ones
\cite{DFS97,FinkelSegur85,DeconinckSwierczewski13}. Our aim throughout is the
development of algorithms capable of dealing with arbitrary compact connected
Riemann surfaces, as is required for the investigation of solutions of, for
instance, the KP equation \cite{DS98,Shiota86}.


%%%%%%%%%%%%%%%%%%%%%%%%%%%%%%%%%%%%%%%%%%%%%%%%%%%%%%%%%%%%%%%%%%%%%%%%%%%%%%%
\section{Finite Genus Solutions}\label{sec:kp-finite-genus-solutions}
%%%%%%%%%%%%%%%%%%%%%%%%%%%%%%%%%%%%%%%%%%%%%%%%%%%%%%%%%%%%%%%%%%%%%%%%%%%%%%%

There is a large class of so-called ``finite-genus'' solutions to the KP
equation. These solutions are quasiperiodic -- functions $f$ such that $f(z + T)
\approx f(z)$ for some quasiperiod $T$ and some meaning of ``approximate''. For
example, a function satisfying $f(z + T) = Cf(z)$ is called geometric
quasiperiodic. Similarly, a function satisfying $f(z + T) = f(z) + C$ is called
arithmetic quasiperiodic. In fact, we have already seen an example of
quasiperiodicity in the Riemann theta function $\theta(z,\Omega)$. The
quasiperiod is $T = m + \Omega n$ for $m,n in \ZZ^g$ and the Riemann theta
function satisfies the functional equation,
\begin{equation}
  \theta(z + m + \Omega n, \Omega)
  =
  e^{-2 \pi i \left( \tfrac{1}{2} n \cdot \Omega n + n \cdot z \right)}
  \theta(z,\Omega).
\end{equation}

In order to define the class of finite-genus solutions to the KP equation we
must first derive some components.

Given a genus $g$ Riemann surface $X$ with normalized basis of holomorphic
one-forms $\mathbf{\omega} = \{ \omega_1, \ldots, \omega_g\}$


\begin{theorem} \label{thm:finite-genus-solutions} The KP equation
\begin{equation} \label{eqn:kp}
  u = c + 2 \partial_x^2 \log \theta\Big( \boldsymbol{U}x +
  \boldsymbol{V}y + \boldsymbol{W}t + \Abel\big(P^\infty,\DivD\big) -
  \RCV(P^\infty), \Omega \Big),
\end{equation}

\end{theorem}


%%%%%%%%%%%%%%%%%%%%%%%%%%%%%%%%%%%%%%%%%%%%%%%%%%%%%%%%%%%%%%%%%%%%%%%%%%%%%%%
\section{Symmetric Period Bases}\label{sec:kp-symmetric-period-bases}
%%%%%%%%%%%%%%%%%%%%%%%%%%%%%%%%%%%%%%%%%%%%%%%%%%%%%%%%%%%%%%%%%%%%%%%%%%%%%%%

% TODO theorem from Belokolos

The approach discussed above provides solutions to the KP equation but make no
restrictions on smoothness. However, for general solutions of the form
\begin{equation} \label{eqn:kp}
  u = c + 2 \partial_x^2 \log \theta\Big(
  \boldsymbol{U}x + \boldsymbol{V}y + \boldsymbol{W}t + \boldsymbol{D},
  \Omega \Big),
\end{equation}
there are certain restrictions imposed on the Riemann surface itself as well as
the choice of phase shift $\boldsymbol{D}$ in order to obtain solutions to the
KP equation that are real and smooth. In this section we will describe a
technique due to Kalla and Klein used to derive such a solution as well as the
algorithmic design in Abelfunctions. Note that this approach makes no direct
contribution to the solution to the initial value problem but is nonetheless an
important perspective of the study of solutions to the KP equation.

%%%%%%%%%%%%%%%%%%%%%%%%%%%%%%% TODO FINISH
We begin with a central theorem.
\begin{theorem} \cite{belokolos1994algebro} For smoothness and reality of
  solutions to the KP equation it is necessary and sufficient that, for a given
  Riemann surface $X$, place at infinity $P_\infty$, and local parameter $p =
  k^{-1}$ at $P_\infty$, that the following conditions are satisfied for the
  vector $\boldsymbol{D}$:
  \begin{itemize}
  \item $X$ admits and antiholomorphic involution $\tau : X \to X, \tau^2 = 1,$
    where $\tau P_\infty = P_\infty$ and $\tau^* k = \bar{k}$.
  \item The set of all fixed ovals of the involution $\tau$ decomposes $X$ into
    two connected components $X^+$ and $X^-$.
  \item If $P_\infty \in $
    
  \end{itemize}

\end{theorem}

The algorithms discussed in Chapter \ref{ch:background} compute the period
matrix of a Riemann surface using the canonical basis of of cycles determined by
the Tretkoff and Tretkoff algorithm \cite{TretkoffTretkoff84}. However, these
cycles may not be invariant under the antiholomorphic involution defined on a
real plane algebraic curve.

\begin{definition} \label{def:topological-type} The {\bf topological} type of a
  real Riemann surface $X$ is given by the tuple $(g, k, a)$ where $g$ is the
  genus of the surface, $k$ is the number of real ovals, and $a = 0$ if the
  curve is dividing and $a = 1$ if the curve is non-dividing. The {\bf real
    ovals} of a surface are the connected components of $X(\mathbb{R})$. $X$ is
  called {\bf dividing} if and only if $X \setminus X(\mathbb{R})$ has two
  connected components. Otherwise, it has one connected component and is called
  {\bf non-dividing}.
\end{definition}

\begin{definition} \label{def:symmetric-homology-basis} Let $\tau$ be an
  antiholomorphic involution on a real Riemann surface $X$. A canonical homology
  basis $(\mathcal{A}, \mathcal{B})$ is called a {\bf symmetric homology basis}
  if it satisfies,
  \begin{equation}
    \begin{pmatrix} \tau \mathcal{A} \\ \tau \mathcal{B} \end{pmatrix}
    =
    \begin{pmatrix}
      \mathbb{I}_g & 0 \\
      \mathbb{H} & -\mathbb{I}_g
    \end{pmatrix}
    \begin{pmatrix} \mathcal{A} \\ \mathcal{B} \end{pmatrix},
  \end{equation}
  where $\mathbb{H}$ is a block diagonal matrix depending on the topological
  type $(g, k, a)$ of $X$ and has one of the following forms:
  \begin{align}
    \mathbb{H} &= \begin{pmatrix}
      0 & 1 &        &   &   &   &        &   \\
      1 & 0 &        &   &   &   &        &   \\
        &   & \ddots &   &   &   &        &   \\
        &   &        & 0 & 1 &   &        &   \\
        &   &        & 1 & 0 &   &        &   \\
        &   &        &   &   & 0 &        &   \\
        &   &        &   &   &   & \ddots &   \\
        &   &        &   &   &   &        & 0
      \end{pmatrix}, & \text{if $k>0$ and $a=0$}, \\
    \mathbb{H} &= \begin{pmatrix}
      \mathbb{I}_{g+1-k} & \\
      & 0
    \end{pmatrix},  & \text{if $k>0$ and $a=1$}, \\
    \mathbb{H} &= \begin{pmatrix}
      0 & 1 &        &   &   \\
      1 & 0 &        &   &   \\
        &   & \ddots &   &   \\
        &   &        & 0 & 1 \\
        &   &        & 1 & 0
      \end{pmatrix}, & \text{if $k=0$ and $g$ is even}, \\
    \mathbb{H} &= \begin{pmatrix}
      0 & 1 &        &   &   &   \\
      1 & 0 &        &   &   &   \\
        &   & \ddots &   &   &   \\
        &   &        & 0 & 1 &   \\
        &   &        & 1 & 0 &   \\
        &   &        &   &   & 0
    \end{pmatrix}, & \text{if $k=0$ and $g$ is odd}.
  \end{align}
\end{definition}

%%%%%%%%%% TODO DISCUSSION OF SYMPLECTIC TRANSFORMATIONS ON HOMOLOGY BASES
%%%%%%%%%% cf. p30 of belokolos
Recall that 

The goal is to construct a symplectic transformation
\begin{equation}
  \Gamma = \begin{pmatrix} A & B \\ C & D \end{pmatrix}
\end{equation}
mapping an arbitrary canonical homology basis $(\tilde{\mathcal{A}},
\tilde{\mathcal{B}})$ to an equivalent, in the sense of Siegel, symmetric
homology basis $(\mathcal{A}, \mathcal{B})$ satisfying,
\begin{equation}
  \begin{pmatrix} \tau \mathcal{A} \\ \tau \mathcal{B} \end{pmatrix}
  \begin{pmatrix}
      \mathbb{I}_g & 0 \\
      \mathbb{H} & -\mathbb{I}_g
  \end{pmatrix}
  \begin{pmatrix} \mathcal{A} \\ \mathcal{B} \end{pmatrix},
\end{equation}
and, therefore, the necessary and sufficient conditions to produce real and
smooth solutions to the KP equation. Vinnikov proves \cite{vinnikov1993self}
that such a symplectic transformation exists,
\begin{equation}
  \begin{pmatrix} A & B \\ C & D \end{pmatrix}
  \begin{pmatrix} \tau{\mathcal{A}} \\ \tau{\mathcal{B}} \end{pmatrix} = 
  \begin{pmatrix} \mathcal{A} \\ \mathcal{B} \end{pmatrix}.
\end{equation}

We summarize the work of Kalla and Klein

%%%%%%%%%%%%%%%%%%%%%%%%%%%%%%%%%%%%%%%%%%%%%%%%%%%%%%%%%%%%%%%%%%%%%%%%%%%%%%%
\section{Computations and Results}\label{sec:kp-computations-and-results}
%%%%%%%%%%%%%%%%%%%%%%%%%%%%%%%%%%%%%%%%%%%%%%%%%%%%%%%%%%%%%%%%%%%%%%%%%%%%%%%

%%%%%%%%%%%%%%%%%%%%%%%%%%%%%%%%%%%%%%%%%%%%%%%%%%%%%%%%%%%%%%%%%%%%%%%%%%%%%%%
\section{Future Work}\label{sec:kp-future-work}
%%%%%%%%%%%%%%%%%%%%%%%%%%%%%%%%%%%%%%%%%%%%%%%%%%%%%%%%%%%%%%%%%%%%%%%%%%%%%%%