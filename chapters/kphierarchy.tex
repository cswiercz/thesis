\chapter{The KP Hierarchy and Algebraic Curves}

In this chapter we establish the primary connection between the KP equation and
Riemann surfaces. We begin by constructing the KP hierarchy, a system of
infinitely many linear evolution equations for a wave function $\Psi =
\Psi(t_1,t_2,\ldots)$ in infinitely many variables that, in a sense,
generalizes the Lax pair formulation [XXX]. We then examine what
happens when we consider solutions which are stationary with respect to two of
these variables, $t_r$ and $t_n$. Introducing these stationary flows produces
an algebraic curve or, more generally, a Riemann surface. Finally, we briefly
describe how certain properties of this Riemann surface can lead to a solution
to the corresponding KP hierarchy solution.

\section{Constructing the KP Hierarchy} \label{sec:2.1}

In several circles, the KP hierarchy is generated from physical considerations
[XXX]. In this section we present a pruely algebraic approach to generating the
hierarchy using the work of Gel'fand and Dikii [XXX], Adler [XXX], Strampp and
Oevel [XXX]. Deconinck in his thesis provides ...[XXX].

Consider the differential operator $\partial = \partial_x$. We define a
pseudo-differential operator to be a Laurent series in $\partial$ with
functions of $x$ as coefficients. For example, the operator
\begin{equation}
  u \partial^{-1}
\end{equation}
acts on a function $f$ by taking its antiderivative and multiplying it by
$u$. Note that the operator
\begin{equation}
  \partial w
\end{equation}
acts on a function $f$ in the following way
\begin{equation}
  (\partial w) f = \partial (w f) = w_x f + w f_x =
  (w_x \partial^0 + w \partial^1) f.
\end{equation}
Note that the expression above can be rewritten such that the
pseudo-differential operators $\partial^n$ appear on the right-hand side of
each term. Throughout this chapter we will chose to normalize a
pseudo-differential operator in this way. This can be done so using the
Leibnitz rule: for any $n \in \ZZ$ and any function $f$
\begin{equation} \label{eqn:leibnitz}
  \partial^n f = \sum_{k=0}^\infty \binom{n}{k} \f^(i)\partial^{n-i}
\end{equation}
where
\begin{equation}
  \binom{n}{k} = \frac{n(n-1) \cdots (n-k+1)}{k!}.
\end{equation}
Note that this definition is valid for negative $n$.

Let $\partial = \partial_x$ and
consider the pseudo-differential operator
\begin{equation}
  L
  =
  \partial + u_2 \partial^{-1} + u_3 \partial_3^{-2} + \cdots
  =
  \sum_{j=-\infty}^1 u_{1-j} \partial^j,
\end{equation}
where $u_0 \equiv 0$ and $u_1 \equiv 1$ and $u_j = u_j(x)$. 
