%%%%%%%%%%%%%%%%%%%%%%%%%%%%%%%%%%%%%%%%%%%%%%%%%%%%%%%%%%%%%%%%%%%%%%%%%%%%%%%
%%%%%%%%%%%%%%%%%%%%%%%%%%%%%%%%%%%%%%%%%%%%%%%%%%%%%%%%%%%%%%%%%%%%%%%%%%%%%%% 
\chapter{Abelfunctions}
%%%%%%%%%%%%%%%%%%%%%%%%%%%%%%%%%%%%%%%%%%%%%%%%%%%%%%%%%%%%%%%%%%%%%%%%%%%%%%%
%%%%%%%%%%%%%%%%%%%%%%%%%%%%%%%%%%%%%%%%%%%%%%%%%%%%%%%%%%%%%%%%%%%%%%%%%%%%%%%

The primary result of this thesis is the Sage software library Abelfunctions
\cite{abelfunctions}. The goal of Abelfunctions is to provide general and easy
to use framework for computing with Abelian functions, Riemann surfaces, and
algebraic curves.

Abelfunctions was inspired by the Maple software package Algcurves developed by
Deconinck, van Hoeij, and Patterson \cite{algcurves}. The need for Abelfunctions
grew out of the difficulty in keeping Algcurves up to date within Maple's
proprietary software. Because of the need to resolve issues as they are
encountered as well as allowing the ability to improve upon the software when
new algorithms are developed, Abelfunctions was created using a completely
open-source model within completely open-source software.

This software can be obtained on Github at
\url{http://github.com/abelfunctions}. Developers are encouraged to examine
Appendix [XXX] where a description of the implementation details of version
[XXX] are provided; the most recent version of the software at the time of this
writing.

%%%%%%%%%%%%%%%%%%%%%%%%%%%%%%%%%%%%%%%%%%%%%%%%%%%%%%%%%%%%%%%%%%%%%%%%%%%%%%%
\section{A Tour of
  Abelfunctions}\label{sec:abelfunctions-a-tour-of-abelfunctions}
%%%%%%%%%%%%%%%%%%%%%%%%%%%%%%%%%%%%%%%%%%%%%%%%%%%%%%%%%%%%%%%%%%%%%%%%%%%%%%%

\begin{quote}
  In this section we demonstrate the various capabilities of the Abelfunctions
  library. The full source code of this demonstration can be found at
  \url{https://github.com/cswiercz/thesis} in the {\tt code} directory as a
  Jupyter notebook.
\end{quote}

Abelfunctions several top-level functions and objects. The user takes advantage
of the Abelfunctions functionality primarily through these objects.
\begin{itemize}
  \item {\tt RiemannSurface} - principle object, defines a Riemann surface given
    a bivariate Sage polynomial.
  \item {\tt AbelMap} - defines the Abel map function from one place on a
    Riemann surface to another. Can accept divisors as well.
  \item {\tt RiemannConstantVector} - defined the Riemann constant vector
    function as a function of an arbitrary place on a Riemann surface.
  \item {\tt RiemannTheta} - the Riemann theta function as a function from
    $\CC^g \times \hg$ to $\CC$.
\end{itemize}

\noindent We begin by importing this functionality and constructing a Riemann
surface corresponding to the plane algebraic curve,

\[
  C : x^2y^3 - x^4 + 1 = 0, \quad x,y \in \CC^*.
\]

\begin{lstlisting}[language=Sage]
sage: from abelfunctions import *  # import the main Abelfunctions functionality
sage: R.<x,y> = QQ[]  # construct a Sage polynomial ring and the above curve
sage: f = y**3 - 2*x**3*y + x**7
sage: X = RiemannSurface(f); X  # construct the corresponding Riemann surface
Riemann surface defined by f = x^7 - 2*x^3*y + y^3
\end{lstlisting}

\noindent The genus of the curve is determined using the singularity structure.
In this case, the Riemann surface is of genus four.

\begin{lstlisting}[language=Sage]
sage: g = X.genus(); g
4
\end{lstlisting}

\noindent We can verify this with the singularity data.

\begin{lstlisting}[language=Sage]
sage: from abelfunctions.singularities import singularities
sage: for point, (m, delta, r) in singularities(f):
 ...:     print point, '\tdelta invariant =', delta
(0, 1, 0)    delta invariant = 2
sage: d = f.total_degree()
sage: g = (d-1)*(d-2)/2 - 2  # (-2) from the delta inv. above
sage: g
4
\end{lstlisting}

%------------------------------------------------------------------------------
\subsection{Places and Divisors}
% ------------------------------------------------------------------------------

The curve $C$ has the branch points,

\begin{lstlisting}[language=Sage]
sage: b = X.branch_points; b
[-1*I, -1, 0, 1, 1*I, +Infinity]
\end{lstlisting}

\noindent This implies that at least one of the places above each of these
points is ramified. In fact, in this example each branch point has a
fully-ramified place lying above it on the Riemann surface.

\begin{lstlisting}[language=Sage]
sage: places = X(0)
sage: for P in places: print P
(-t^3, -t^-2 + O(t^0))
sage: P.puiseux_series.extend(16); P
(-t^3, -t^-2 + 1/3*t^10 + 1/9*t^22 + O(t^31))
\end{lstlisting}


%%%%%%%%%%%%%%%%%%%%%%%%%%%%%%%%%%%%%%%%%%%%%%%%%%%%%%%%%%%%%%%%%%%%%%%%%%%%%%%
\section{Improvements on
  Algorithms}\label{sec:abelfunctions-improvements-on-algorithms}
%%%%%%%%%%%%%%%%%%%%%%%%%%%%%%%%%%%%%%%%%%%%%%%%%%%%%%%%%%%%%%%%%%%%%%%%%%%%%%%


%%%%%%%%%%%%%%%%%%%%%%%%%%%%%%%%%%%%%%%%%%%%%%%%%%%%%%%%%%%%%%%%%%%%%%%%%%%%%%%
\subsection{Analytic
  Continuation}\label{subsec:abelfunctions-analytic-continuation}
%%%%%%%%%%%%%%%%%%%%%%%%%%%%%%%%%%%%%%%%%%%%%%%%%%%%%%%%%%%%%%%%%%%%%%%%%%%%%%%


%%%%%%%%%%%%%%%%%%%%%%%%%%%%%%%%%%%%%%%%%%%%%%%%%%%%%%%%%%%%%%%%%%%%%%%%%%%%%%%
\subsection{Integration}\label{subsec:abelfunctions-integration}
%%%%%%%%%%%%%%%%%%%%%%%%%%%%%%%%%%%%%%%%%%%%%%%%%%%%%%%%%%%%%%%%%%%%%%%%%%%%%%%


%%%%%%%%%%%%%%%%%%%%%%%%%%%%%%%%%%%%%%%%%%%%%%%%%%%%%%%%%%%%%%%%%%%%%%%%%%%%%%%%
\subsection{Riemann Constant
  Vector}\label{subsec:abelfunctions-riemann-constant-vector}
%%%%%%%%%%%%%%%%%%%%%%%%%%%%%%%%%%%%%%%%%%%%%%%%%%%%%%%%%%%%%%%%%%%%%%%%%%%%%%%
%%% Local Variables:
%%% mode: latex
%%% TeX-master: "../thesis"
%%% End:
